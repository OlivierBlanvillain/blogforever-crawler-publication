\section{Introduction}

The blogosphere is an established channel of online communication which 
bears great significance~\cite{oevre2012}. Wordpress, a single blog 
publishing company, reports more than 1 million new posts and 1.5 million 
new comments each day~\cite{wordpress2014}. These overwhelming numbers 
illustrate the importance of blogs in most aspects of private and 
business life~\cite{chung2007uses}. Blogs contain data with historic, 
political, social and scientific value which need to be accessible 
for current and future generations. For instance, blogs proved to 
be an important resource during the 2011 Egyptian revolution by 
playing an instrumental role in the organization and implementation of 
protests~\cite{nahedeltantawy2012}. The problem is that blogs disappear 
every day~\cite{johnson2008blogs} because there is no standard method or 
authority to ensure blog archiving and long-term digital preservation.

Among the challenges in developing a blog archiving software
is the design of a web crawler capable of efficiently traversing blogs 
to harvest their content. The sheer size of the blogosphere combined with 
an unpredictable publishing rate of new information call for a highly 
scalable system, while the lack of programmatic access to the complete 
blog content makes the use of automatic extraction techniques necessary. 
The variety of available blog publishing platforms offers a limited 
common set of properties that a crawler can exploit, further narrowed by 
the ever-changing structure of blog contents. Finally, an increasing 
number of blogs heavily rely on dynamically created content to present 
information, using the latest web technologies, hence invalidating 
traditional web crawling techniques.

A key characteristic of blogs which differentiates them from regular 
websites is their association with web feeds~\cite{lindahl2003weblogs}. 
Their primary use is to provide a uniform subscription mechanism, thereby 
allowing users to keep track of the latest updates without the need to 
actually visit blogs. Concretely, a web feed is an XML file containing 
links to the latest blog posts along with their articles (abstract or 
full text) and associated metadata~\cite{board2007rss}. While web feeds 
essentially solve the question of update monitoring, their limited size 
makes it necessary to download blog pages to harvest previous content.

This paper presents the latest developments of the open-source 
BlogForever Crawler, a key component of the BlogForever 
platform~\cite{kasioumis2013towards} responsible for traversing blogs, 
extracting their content and monitoring their updates. Our main 
objectives in this work are to introduce a new approach to blog data 
extraction and to present the architecture and implementation of a blog 
crawler capable of extracting articles, authors, publication dates, 
comments and potentially any other element which appear in weblog
web feeds. Our contributions can be summarized as follows:

\begin{itemize}
\item 
A new algorithm to build extraction rules from web feeds and an optimised 
reformulation based on a particular string similarity algorithm featuring
linear time complexity.
\item 
A methodology to use the algorithm for blog article extraction and how 
it can be augmented to be used with other blog elements such as authors, 
publication dates and comments.
\item 
The overall BlogForever crawler architecture and implementation with a 
focus on design decisions, modularity, scalability and interoperability.
\item 
An approach to use a complete web browser to render JavaScript powered 
web pages before processing them. This step allows our crawler to 
effectively harvest blogs built with modern technologies, such as 
the increasingly popular third-party commenting systems.
\item 
A mapping of the extracted blog content to Archival Information Packages 
(AIPs) using METS and MARCXML standards for interoperability purposes.
\item 
An evaluation of the content extraction and execution time of our 
algorithm against three state-of-the-art web article extraction algorithms.\\
\end{itemize}

The concepts emerging from our research are viewed in the context of the 
BlogForever platform but the presented algorithms, techniques and system 
architectures can be used in other applications related to Wrapper 
Generation and Web Data Extraction.

The rest of this work is structured as follows: Section 2 presents 
related work. Section 3 introduces the new algorithms to extract data 
from blogs. Section 4 presents the blog crawler system architecture and 
implementation. Section 5 presents the evaluation and results. Finally, 
our conclusions and some discussion on our work are presented in section 6.
