\section{Conclusion and Future Work}

In this paper, we presented the internals of the BlogForever web crawler. Its central article extraction procedure based on extraction rules generation was introduced along with theoretical and empirical evidence validating the approach. A simple adaptation of this procedure that allows to extract different types of content, including authors, dates and comments was then presented. In order to support rapidly evolving web technologies such as JavaScript-generated content, the crawler uses a web browser to render pages before processing them. We also discussed the overall software architecture, highlighting the design choices made to achieve both modularity and scalability. Finally, we evaluated our content extraction algorithm against three state-of-the-art web article extraction algorithms.

Future work could investigate \emph{hybrid} extraction algorithms to try and achieve near 100\% success rates. 

\TODO{\\
  > You didn't say anything about this in the evaluation, did you?
  Either present some numbers about it or remove this thing from the conclusions.
  \\\\
  I think I will simply rephrase it with less precisions, but the idea is that there is some overlap between the 7\% of posts we fail to extract and the 88.1\% of posts extracted by readability, likely in the order of 7\% * 88.1\% as the conditions of failure are completely independent.
}

Even if the overall performance of our approach is better than those of comparable projects (as shown in the evaluation), there are still a few cases where other techniques managed to extract data and we did not. This suggests that combining our approach with others such such as word density, tree edit distance matching or even spacial reasoning could lead to better performance.

Another possible research direction would be the deployment of the BlogForever crawler on a large scale distributed system. This is particularly relevant in the domain of web crawling given that intensive network operations can be a serious bottleneck. Crawler greatly benefits from the use of multiple Internet access points which makes them natural candidates for distributed computing. We intend to explore these opportunities in our future work.
