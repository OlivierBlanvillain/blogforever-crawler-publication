\section{Conclusion and Future Work}
% What we presented, what we solved
In this paper, we presented the internals of the BlogForever web crawler\comment{maybe rephrase, since this is not the current one, but a suggestion for an open source one? See how we introduced it, ask Vangelis.}. Its central article extraction procedure based on extraction rules generation was introduced along with theoretical\comment{evidence can't be theoritical. Maybe theoritical basis or explanation?}\TODO{I thought it was a synonym for proof, \surl{http://goo.gl/DgUIX8} and \surl{http://goo.gl/Icqd2x}} and empirical evidence validating the approach. A simple adaptation of this procedure that allows to extract different types of content, including authors, dates and comments was then presented. In order to support rapidly evolving web technologies such as JavaScript-generated content, the crawler uses a web browser to render pages before processing them. We also discussed the overall software architecture, highlighting the design choices made to achieve both modularity and scalability.\TODO{Finally, ... evaluation}

% Future work, hybrid algos
Future work could investigate \emph{hybrid} extraction algorithms to try and achieve near 100\% success rates. Even if the overall performance of our approach is better than those of comparable projects as shown in \autoref{evaluation}\TODO{It think it might be better to avoid references in abstract and conclusion as a lot of readers will skip everything else. Maybe "as shown in the Evaluation"?}, there are still a few cases where other techniques managed to extract data and we did not. This suggests that combining our approach with others such such as word density, tree edit distance matching or even spacial reasoning could lead to better performance.

% Deployment on a distributed architecture
Another possible research direction would be the deployment of the BlogForever crawler on a large scale distributed system. This is particularly relevant in the domain of web crawling given that intensive network operations can be a serious bottleneck. The which benefits from the use of multiple Internet access points\comment{it sounds like the bottleneck benefits from the use of... --> rephrase, let's discuss}. We intend to explore this opportunities in our future work.

% Low level PhantomJS hooks
% I guess two future works is enough...

% Conclude the conclude
To open up further possibilities\comment{let's rephrase, can't think of something now.}, the crawler presented in this paper is available under the MIT license from the project's website \cite{blogforevercrawler}. \comment{why not put the website directly here?}
