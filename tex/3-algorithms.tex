\section{Algorithms}

This section explains in details the algorithm we developed to extract blog posts content and it's variations for authors, date and comments. We will see how blog specific characteristics can be exploited to obtain an extraction procedure applicable to all posts of a blog. A careful attention will be paid to minimizing algorithmic complexity while keeping the simple and general approach of the algorithm.


%%%%%%%%%%%%%%%%%%%%%%%
\subsection{Motivation}
- getting content from HTML is not easy.
\cite{worldwidewebconsortiumw3c2002}

% Assumptions when working with blogs
\Anotecontent{assumptions}{We observed during evaluation that all failing test blogs where violating one of these assumptions.}
Working with blogs allows to make assumptions\Anote{assumptions} that are central in our extraction procedure:
\begin{enumerate}
  \item Web feeds provide a structured and standardized view of the most recent posts
  \item Posts of a same blog share the similar HTMP structure.
\end{enumerate}

% Web feeds limitations
In average, web feeds only contain about 20 entries \cite{french paper} which is generally lower than the total number of posts in a blog. In order to effectively archive old content from blogs it is necessary to download and process pages beyond the one referenced by the feed. The algorithm we are going to present in this section as for function to build wrappers for the relevant data of blog posts, which are going to be use

% Per blog procedure, inputs/output
Each time a new crawl is initiated our system builds a set of wrappers that are then used for the entirety of the blog. With this wrappers in hands, extraction data from the HTML page of a blog post is nothing more than parsing the HTML page running each XPath query, which can all by done with the standard python library \footnote{lxml, beautifullsoop}.


%%%%%%%%%%%%%%%%%%%%%%%%%%%%%%%
\subsection{Content extraction}
- related algorithms, general brute force viewpoint
- algorithm
- XPath selectors (id, else class, else path)
- string similarity
- caching


%%%%%%%%%%%%%%%%%%%%%%%%%%%%%%%%%%%%%%%%%%%%%%%%%%%%%%%%%%%%%%%%%
\subsection{Variations for comments, date and authors extraction}
- comments
- date authors
