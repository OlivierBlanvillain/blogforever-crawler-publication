\documentclass{acm_proc_article-sp}

\usepackage{epsfig}
\usepackage[utf8]{inputenc}
\usepackage[T1]{fontenc}
\usepackage{url}
\usepackage{hyperref}

\newcommand{\TODO}[1]{\emph{\color{red} TODO: #1}}
\newcommand{\surl}[1]{\urlstyle{ff}\url{#1}}

\begin{document}

\title{Web feed based content extraction with JavaScript rendering for the Invenio crawler}

\numberofauthors{3}

\author{
% 1st. author
\alignauthor
Olivier Blanvillain\\
       \affaddr{École polytechnique fédérale de Lausanne (EPFL)}\\
       \affaddr{1014 Lausanne, Switzerland}\\
       \email{olivier.blanvillain@epfl.ch}
% 2nd. author
\alignauthor
Nikos Kasioumis\\
       \affaddr{European Organization for Nuclear Research (CERN)}\\
       \affaddr{1211 Geneva 23, Switzerland}\\
       \email{nikos.kasioumis@cern.ch}
% 3rd. author
\alignauthor
Vangelis Banos\\
       \affaddr{Department of Informatics}\\
       \affaddr{Aristotle University of Thessaloniki, Greece}\\
       \email{vbanos@gmail.com}
}

\maketitle
\begin{abstract}

Blogs are one of the most prominent means of communication on the web. 
Their content, interconnections and influence constitute a unique 
socio-technical artefact of our times which needs to be preserved. 
The BlogForever project has established best practices and developed an 
innovative system to harvest, preserve, manage and reuse blog content. 
This paper presents the latest developments of the blog crawler which 
is a key component of the BlogForever platform. More precisely, our 
work concentrates on techniques to automatically extract content such 
as articles, authors, dates and comments from blog posts. To achieve 
this goal, we introduce a simple yet robust and scalable algorithm to 
generate extraction rules based on string matching using the blog's 
web feed in conjunction with blog hypertext. Furthermore, we present a 
system architecture which is characterised by efficiency, modularity, 
scalability and interoperability with third-party systems. Finally, 
we conduct thorough evaluations of the performance and accuracy of
our system.


\end{abstract}

% A category with the (minimum) three required fields
\category{H.4}{Information Systems Applications}{Miscellaneous}
%A category including the fourth, optional field follows...
\category{D.2.8}{Software Engineering}{Metrics}[complexity measures, performance measures]

\terms{Theory}

\keywords{ACM proceedings, \LaTeX, text tagging} % NOT required for Proceedings

\section{Intro}

\begin{enumerate}
  \item Introduce the importance for blog preservation
  \item Explain the difficulties in harvesting blogs
  \item Explain why open source and Invenio
  \item Introduce the blog crawler
\end{enumerate}

Challenges consist of:
\begin{enumerate}
  \item Providing a high degree of automation
  \item Dealing with large volumes of data
\end{enumerate}

\comment{general introduction to web archiving and web crawlers}
Web archiving is the process of harvesting and gathering web content in order to safely preserve it for posterity.
As the volume and importance of the information on the World Wide Web increases, web archiving becomes more and more relevant and its importance becomes clearer.\\
Web crawlers are an essential part of web archiving, allowing for automated capturing of these resources.\\
An efficient and accurate web crawler is the first crucial step to successful digital preservation.\\
\\
\comment{introduction to blogs, their nature and their significance}
An estimate xx\% of current web content is blogs (reference). OR There is an estimate of xxxxx blogs online currently.\\
A blog is a \emph{frequently updated web site consisting of personal observations, excerpts from other sources, etc., typically run by a single person, and usually with hyperlinks to other sites; an online journal or diary.}\\
\comment{http://www.oed.com/view/Entry/256743\#eid11173991}
Given the nature of blogs, they are expectedly dynamic and highly volatile web resources.\\
\\
Blogs are nowadays a popular means of communication and expression of ideas and have been adopted by universities, institutions and scolars (reference?).\\
Blogs have recenlty been used to disseminate ideas in times of political turmoil or even war. (examples, Egypt, Syria etc)\\
Their political, scientific and cultural significance is undisputed. (rephrase to avoid need for reference?)\\
\\
However, to this day there is no standard method or authority that ensures blog archiving and xx\% dissapear every day (reference).\\
\comment{A Tale of a Disappearing Website : http://blogs.loc.gov/digitalpreservation/2012/01/a-tale-of-a-disappearing-website/}
\comment{Considerations for the Preservation of Blogs : http://www.digitalpreservationeurope.eu/publications/briefs/preservartion\_blogs.pdf}
Therefore, it is crucial that the necessary foundations for blog archiving are put in place.\\
\\
\comment{present background in blog archiving and constraints}
The blogosphere is a massive web resource of unstructured data.\\
Identifying, harvesting and parsing that data can be a very challenging task.\\
Challenges and constraints: size of blogosphere, heterogeneous nature of different publishing platforms, ever-changing structure, dynamically created content using modern web technologies.\\
complexity in parsing blogs from unstructured data to structured information.\\
programmatic access to blog content unavailable\\
unpredictable publishing rate --> scalability\\
\\
\comment{present issues in current solutions}
Is this needed? Can write in "Related Work" instead.
\\
\comment{present our main ideas and principles}
Web feeds\\
HTML + XPATH\\
efficient string similarity\\
JavaScript\\
\\
\comment{present our statement and target}
improve efficiency, accuracy and general quality to help with archiving and understanding, using (information retrieval).
\\
\comment{present the contents of the paper}
In chapter 1, chapter 2....
\\
Notes:\\
\\
(from wikipedia) The largest web archiving organization based on a bulk crawling approach is the Internet Archive which strives to maintain an archive of the entire Web. National libraries, national archives and various consortia of organizations are also involved in archiving culturally important Web content. Commercial web archiving software and services are also available to organizations who need to archive their own web content for corporate heritage, regulatory, or legal purposes.
\\
\\
\comment{Aims and contributions of this work. Example from my last paper to see the format:\\
http://purl.pt/24107/1/iPres2013\_PDF/\\
CLEAR\%20a\%20credible\%20method\%20to\%20\\
evaluate\%20website\%20archivability.pdf}
\comment{Semi-structured nature of Web pages\\
Labeled ordered rooted trees\\
XML Path Language(XPath)}

\section{Related work}
2-related-work.tex

Web Object Identification for Web Automation and Meta-Search,
\surl{http://www.dbai.tuwien.ac.at/proj/tamcrow/download/Kordomatis2013WIMS.pdf}

Self-supervised Automated Wrapper Generation for Weblog Data Extraction,
\surl{https://github.com/OlivierBlanvillain/blogforever-crawler-publication/raw/master/papers/bncod_published.pdf}

Web Data Extraction, Applications and Techniques: A Survey,
\surl{http://www.emilio.ferrara.name/wp-content/uploads/2011/07/survey-csur.pdf}

Archiving Data Objects using Web Feeds,
\surl{http://hal.archives-ouvertes.fr/docs/00/53/79/62/PDF/iwawienna.pdf}

Intelligent and Adaptive Crawling of Web Applications for Web Archiving,
\surl{http://pierre.senellart.com/publications/faheem2013intelligent.pdf}

Zero-cost Labelling with Web Feeds for Weblog Data Extraction,
\surl{http://www2013.org/companion/p73.pdf}

OXPath: A Language for Scalable, Memory-efficient Data Extraction from Web Applications,
\surl{http://www.vldb.org/pvldb/vol4/p1016-furche.pdf}

\section{Architecture}

\subsection{Scrapy and Invenio}
Introduce both projects with their respective strengths, tool/technology justification.

\subsection{JavaScript Rendering}
\begin{enumerate}
  \item PhantomJS
  \item "Show more comments" click
  \item Done upfront, every thing else is independent
\end{enumerate}

\subsection{Extraction Process}
High level overview: XPath build from RSS feeds

\section{Algorithms}

\subsection{Content extraction}
\begin{enumerate}
  \item The algorithm
  \item Choice of selectors (id, else class)
  \item Choice of string similarity
\end{enumerate}

\subsection{Variations for comments and date}
All the details about how to use the same algorithm to extract comments and dates.

\section{Evaluation}

\subsection{Introcution}
What is evaluated (not comments, iframes, authors, dates...)

\subsection{Content extraction}
\begin{enumerate}
  \item running time
  \item quality of output compared
\end{enumerate}

\subsection{JavaScript rendering}
\begin{enumerate}
  \item running time
\end{enumerate}

\section{Conclusion and Future Work}
% What we presented, what we solved
In this paper, we presented the internals of the BlogForever web crawler. Its central article extraction procedure based on extraction rules generation was introduced along with theoretical and empirical evidences validating the approach. Simple adaptation of this procedure allowed to extract different types of contents, including authors, dates and comments. In order to support rapidly evolving web technologies such as JavaScript generated content, the crawler uses a web browser to render pages before processing. We also discussed the overall software architecture, highlighting the design choices made to achieve both modularity and scalability.

% Future work, hybrid algos
Future work could investigate \emph{hybrid} extraction algorithms to try to achieve near 100\% success rates. Even if overall our approach obtains better performs, there are few cases where other techniques managed to extract data where we did not. This suggests that combining our approach with others such such as word density, tree edit distance matching or even spacial reasoning could lead to better performances.

% Deployment on a distributed architecture
Another possible research direction would be the deployment of the BlogForever crawler on a large scale distributed system. This is particularly relevant in the domain of web crawling given that intensive network operations can be a serious bottleneck which benefits from the use of multiple Internet access points. We intend to explore this opportunities in our future work.

% Low level PhantomJS hooks
% I guess two future works is enough...

% Conclude the conclude
To open up further possibilities, the crawler presented in this paper is available under the MIT license from the project's website \cite{blogforevercrawler}.


\bibliographystyle{abbrv}
\bibliography{sigproc}
\end{document}
