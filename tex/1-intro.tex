\section{Intro}

\begin{enumerate}
  \item Introduce the importance for blog preservation
  \item Explain the difficulties in harvesting blogs
  \item Explain why open source and Invenio
  \item Introduce the blog crawler
\end{enumerate}

Challenges consist of:
\begin{enumerate}
  \item Providing a high degree of automation
  \item Dealing with large volumes of data
\end{enumerate}

\comment{general introduction to web archiving and web crawlers}
Web archiving is the process of harvesting and gathering web content in order to safely preserve it for posterity.
As the volume and importance of the information on the World Wide Web increases, web archiving becomes more and more relevant and its importance becomes clearer.\\
Web crawlers are an essential part of web archiving, allowing for automated capturing of these resources.\\
An efficient and accurate web crawler is the first crucial step to successful digital preservation.\\
\\
\comment{introduction to blogs, their nature and their significance}
An estimate xx\% of current web content is blogs (reference). OR There is an estimate of xxxxx blogs online currently.\\
A blog is a \emph{frequently updated web site consisting of personal observations, excerpts from other sources, etc., typically run by a single person, and usually with hyperlinks to other sites; an online journal or diary.}\\
\comment{http://www.oed.com/view/Entry/256743\#eid11173991}
Given the nature of blogs, they are expectedly dynamic and highly volatile web resources.\\
\\
Blogs are nowadays a popular means of communication and expression of ideas and have been adopted by universities, institutions and scolars (reference?).\\
Blogs have recenlty been used to disseminate ideas in times of political turmoil or even war. (examples, Egypt, Syria etc)\\
Their political, scientific and cultural significance is undisputed. (rephrase to avoid need for reference?)\\
\\
However, to this day there is no standard method or authority that ensures blog archiving and xx\% dissapear every day (reference).\\
\comment{A Tale of a Disappearing Website : http://blogs.loc.gov/digitalpreservation/2012/01/a-tale-of-a-disappearing-website/}
\comment{Considerations for the Preservation of Blogs : http://www.digitalpreservationeurope.eu/publications/briefs/preservartion\_blogs.pdf}
Therefore, it is crucial that the necessary foundations for blog archiving are put in place.\\
\\
\comment{present background in blog archiving and constraints}
The blogosphere is a massive web resource of unstructured data.\\
Identifying, harvesting and parsing that data can be a very challenging task.\\
Challenges and constraints: size of blogosphere, heterogeneous nature of different publishing platforms, ever-changing structure, dynamically created content using modern web technologies.\\
complexity in parsing blogs from unstructured data to structured information.\\
programmatic access to blog content unavailable\\
unpredictable publishing rate --> scalability\\
\\
\comment{present issues in current solutions}
Is this needed? Can write in "Related Work" instead.
\\
\comment{present our main ideas and principles}
Web feeds\\
HTML + XPATH\\
efficient string similarity\\
JavaScript\\
\\
\comment{present our statement and target}
improve efficiency, accuracy and general quality to help with archiving and understanding, using (information retrieval).
\\
\comment{present the contents of the paper}
In chapter 1, chapter 2....
\\
Notes:\\
\\
(from wikipedia) The largest web archiving organization based on a bulk crawling approach is the Internet Archive which strives to maintain an archive of the entire Web. National libraries, national archives and various consortia of organizations are also involved in archiving culturally important Web content. Commercial web archiving software and services are also available to organizations who need to archive their own web content for corporate heritage, regulatory, or legal purposes.
\\
\\
\comment{Aims and contributions of this work. Example from my last paper to see the format:\\
http://purl.pt/24107/1/iPres2013\_PDF/\\
CLEAR\%20a\%20credible\%20method\%20to\%20\\
evaluate\%20website\%20archivability.pdf}
\comment{Semi-structured nature of Web pages\\
Labeled ordered rooted trees\\
XML Path Language(XPath)}
