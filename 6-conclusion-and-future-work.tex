\section{Discussion and Conclusions}
\label{conclusionandfuturework}
In this article, we presented our extended work on a scalable approach to 
harvest modern weblogs. Our approach is based on a new algorithm to build 
extraction rules from web feeds. The key observations which led us to 
the inception and implementation of this algorithm are the facts that: 
(1) weblogs provide web feeds which include structured and standardized 
views of the latest blog posts, and (2) post of the same weblog share 
a similar HTML structure. Following, we presented a simple adaptation 
of this procedure that allows extracting different types of content, 
including authors, dates, comments and potentially any other element. 
The elaboration of this process enables the wider use of this method 
as it is feasible to devise variations of the proposed method to extract 
any kind of weblog content.

A critical part of this work was the presentation of the BlogForever 
crawler architecture and discussion of the software tools and the novel 
techniques we used. In order to support rapidly evolving web technologies 
such as JavaScript-generated content, the crawler uses a headless web 
browser to render pages before processing them. Another important aspect 
of the crawler architecture was the interoperability provisions. 
In our design, we introduced a new metadata schema for interoperability, 
encoding weblog crawling results in Archival Information Packages (AIPs) 
using established open standards. This feature enables the use of our 
system in many contexts and with multiple different back-ends, increasing 
its relevance and reusability. We also highlighted the design choices 
made to achieve both modularity and scalability. This is particularly 
relevant in the domain of web crawling given that intensive network 
operations can be a serious bottleneck. Crawlers greatly benefit from 
the use of multiple Internet access points which makes them natural 
candidates for distributed computing.

Our method had great success with content extraction accuracy and 
performance against state-of-the-art open source web article extraction 
systems as presented in the evaluation section. We have to note thought 
that our experiments on a considerably large weblogs dataset showed that 
there were some failing tests which stem from either the violation of one 
of our two key observations, or from an insufficient amount of text in 
posts. Therefore, it is suggested to potential users to ensure that these 
observations are valid on the target weblogs before proceeding with using 
the BlogForever crawler. Future work could attempt to alleviate this problem
using \emph{hybrid} extraction algorithms. Combining our approach with
others techniques such as word density or special reasoning could lead
to better performance given that these techniques are insensible to the
above issues.
